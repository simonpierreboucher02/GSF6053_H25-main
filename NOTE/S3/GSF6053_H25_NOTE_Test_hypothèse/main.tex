\documentclass[14pt]{extarticle} % Utilisation de extarticle pour supporter 14pt

% ------------------------------------------------------------------------
% Packages indispensables et recommandés
% ------------------------------------------------------------------------
\usepackage[utf8]{inputenc}     % Encodage des caractères
\usepackage[T1]{fontenc}        % Encodage de la police
\usepackage[french]{babel}      % Support de la langue française
\usepackage{amsmath, amssymb}   % Environnements mathématiques enrichis
\usepackage{amsthm}             % Environnements théorèmes
\usepackage{graphicx}           % Inclusion d'images
\usepackage{hyperref}           % Liens hypertextes
\usepackage{geometry}           % Gestion des marges
\usepackage{titlesec}           % Personnalisation des titres
\usepackage{setspace}           % Gestion de l'interligne
\usepackage{xcolor}             % Gestion des couleurs
\usepackage{booktabs}           % Pour améliorer les tableaux
\usepackage{fancyhdr}           % Personnalisation des en-têtes et pieds de page
\usepackage{cleveref}           % Références intelligentes
\usepackage{caption}            % Pour personnaliser les légendes
\usepackage{enumitem}           % Pour personnaliser les listes
\usepackage{pdflscape}          % Pour placer les tableaux en paysage

% ------------------------------------------------------------------------
% Configuration de la mise en page
% ------------------------------------------------------------------------
\geometry{
    a4paper,
    margin=25mm
}

% ------------------------------------------------------------------------
% Définition des environnements théorèmes, définitions, etc.
% ------------------------------------------------------------------------
\theoremstyle{definition}
\newtheorem{definition}{Définition}[section]

\theoremstyle{plain}
\newtheorem{proposition}{Proposition}[section]
\newtheorem{theorem}{Théorème}[section]

% ------------------------------------------------------------------------
% Personnalisation des sections
% ------------------------------------------------------------------------
\titleformat{\section}{\Large\bfseries}{\thesection}{1em}{}
\titleformat{\subsection}{\large\bfseries}{\thesubsection}{1em}{}
\titleformat{\subsubsection}{\normalsize\bfseries}{\thesubsubsection}{1em}{}

% ------------------------------------------------------------------------
% Commande pour mettre en avant les livres en bleu
% ------------------------------------------------------------------------
\newcommand{\livre}[1]{\textcolor{blue}{#1}}

% ------------------------------------------------------------------------
% Configuration des hyperliens
% ------------------------------------------------------------------------
\hypersetup{
    colorlinks=true,          % Les liens sont colorés
    linkcolor=blue,           % Couleur des liens internes (TOC, références, etc.)
    urlcolor=blue,            % Couleur des URL
    citecolor=blue,           % Couleur des citations
    filecolor=blue,           % Couleur des liens vers des fichiers
    pdfborder={0 0 0},        % Pas de bordure autour des liens
    breaklinks=true            % Permet les sauts de ligne dans les liens
}

% ------------------------------------------------------------------------
% Configuration des en-têtes et pieds de page
% ------------------------------------------------------------------------
\pagestyle{fancy}
\fancyhf{}
\fancyhead[L]{Tests d'Hypothèses}
\fancyhead[R]{Hiver 2025}
\fancyfoot[C]{\thepage}
\fancyfoot[R]{\includegraphics[height=1cm]{logo_universite_laval.png}} % Ajout du logo ERS en bas à droite

% ------------------------------------------------------------------------
% Informations sur le document
% ------------------------------------------------------------------------
\title{Tests d'Hypothèses}
\author{GSF-6053}
\date{Hiver 2025}

% ------------------------------------------------------------------------
% Début du document
% ------------------------------------------------------------------------
\begin{document}

\maketitle

\tableofcontents
\newpage

% Augmenter l'interligne à 1,5
\onehalfspacing

% ------------------------------------------------------------------------
% SECTION : Économétrie financière I GSF 6053
% ------------------------------------------------------------------------

\section{Introduction}

Ce document présente une étude approfondie des tests d'hypothèses en économétrie financière, en mettant l'accent sur les tests de Student. Il est structuré en trois parties principales : une introduction aux tests d'hypothèses, une application pratique des tests de Student dans le cadre du CAPM, et les solutions détaillées des exercices proposés.

% ------------------------------------------------------------------------
% SECTION 1 : Tests d'Hypothèses
% ------------------------------------------------------------------------
\section{Tests d'Hypothèses}

Les tests d'hypothèses sont des outils fondamentaux en économétrie pour évaluer la validité des suppositions concernant les paramètres d'un modèle. Ils permettent de prendre des décisions éclairées basées sur les données observées. Cette section détaille les étapes essentielles pour réaliser un test d'hypothèse efficace.

\subsection{Introduction aux Tests d'Hypothèses}

Un test d’hypothèse consiste à formuler deux hypothèses contradictoires : l'hypothèse nulle (\(H_0\)) et l'hypothèse alternative (\(H_1\)). Le but est de déterminer si les données disponibles permettent de rejeter \(H_0\) en faveur de \(H_1\).

\subsubsection{Formulation des Hypothèses}

\begin{itemize}
    \item \textbf{Hypothèse nulle (\(H_0\))} : C'est l'hypothèse de base que l'on cherche à tester. Elle représente souvent une situation de statu quo ou une absence d'effet. Par exemple, \(H_0\) pourrait stipuler qu'un paramètre est égal à une valeur spécifique ou qu'il n'a pas d'effet significatif.
    \item \textbf{Hypothèse alternative (\(H_1\))} : Elle représente ce que l'on cherche à prouver. \(H_1\) peut être bilatérale (différent de) ou unilatérale (supérieur à ou inférieur à).
\end{itemize}

\subsubsection{Types de Tests}

\begin{itemize}
    \item \textbf{Test Bilatéral} : Utilisé lorsque l'hypothèse alternative indique une différence dans les deux sens (\( \neq \)).
    \item \textbf{Test Unilatéral} : Utilisé lorsque l'hypothèse alternative spécifie une direction particulière (\( > \) ou \( < \)).
\end{itemize}

\subsection{Étapes d'un Test d'Hypothèse}

Pour effectuer un test d'hypothèse, suivez les étapes suivantes :

\begin{enumerate}[label=\textbf{\arabic*.}]
    \item \textbf{Formulation des Hypothèses} : Définir clairement \(H_0\) et \(H_1\).
    \item \textbf{Choix de la Statistique de Test} : Sélectionner une statistique appropriée (t, F, Wald, LR, LM, etc.) en fonction du modèle et des hypothèses.
    \item \textbf{Détermination du Niveau de Significativité (\(\alpha\))} : Choisir le seuil de probabilité pour rejeter \(H_0\). Les niveaux courants sont 5\%, 1\% ou 10\%.
    \item \textbf{Définition de la Région Critique} : Déterminer les valeurs critiques de la statistique de test qui délimitent la région où \(H_0\) sera rejetée.
    \item \textbf{Calcul de la Statistique de Test} : Utiliser les données de l'échantillon pour calculer la valeur de la statistique de test.
    \item \textbf{Prise de Décision} : Comparer la statistique de test avec les valeurs critiques pour décider de rejeter ou non \(H_0\).
\end{enumerate}

\subsection{Interprétation des Résultats}

\begin{itemize}
    \item \textbf{Rejet de \(H_0\)} : Si la statistique de test tombe dans la région critique, on rejette \(H_0\) en faveur de \(H_1\).
    \item \textbf{Non-rejet de \(H_0\)} : Si la statistique de test ne tombe pas dans la région critique, on ne dispose pas de preuves suffisantes pour rejeter \(H_0\).
\end{itemize}

\subsection{La p-value}

La p-value est une mesure cruciale dans les tests d'hypothèses. Elle représente la probabilité d'observer une statistique de test au moins aussi extrême que celle obtenue, sous l'hypothèse que \(H_0\) est vraie.

\begin{itemize}
    \item \textbf{Interprétation} : 
    \begin{itemize}
        \item Une p-value faible (\(< \alpha\)) indique que l'observation est peu probable sous \(H_0\), suggérant le rejet de \(H_0\).
        \item Une p-value élevée (\(\geq \alpha\)) suggère que l'observation est compatible avec \(H_0\), et donc, on ne rejette pas \(H_0\).
    \end{itemize}
    \item \textbf{Utilisation} : Plutôt que de comparer directement la statistique de test aux valeurs critiques, on peut utiliser la p-value pour évaluer la force des preuves contre \(H_0\).
\end{itemize}

\subsection{Erreur de Type I et Type II}

\begin{itemize}
    \item \textbf{Erreur de Type I (\(\alpha\))} : Rejeter \(H_0\) alors qu'il est vrai.
    \item \textbf{Erreur de Type II (\(\beta\))} : Ne pas rejeter \(H_0\) alors qu'il est faux.
\end{itemize}

\subsection{Puissance du Test}

La puissance d'un test est la probabilité de rejeter \(H_0\) lorsqu'il est effectivement faux (\(1 - \beta\)). Une puissance élevée est souhaitable car elle indique une forte capacité à détecter un effet lorsque celui-ci existe.

% ------------------------------------------------------------------------
% SECTION 2 : Application : Tests de Student
% ------------------------------------------------------------------------
\section{Application : Tests de Student}

Les tests de Student, ou tests t, sont couramment utilisés pour évaluer la significativité des coefficients dans les modèles de régression linéaire. Ils permettent de déterminer si un paramètre est statistiquement différent d'une valeur hypothétique (généralement zéro).

\subsection{Contexte de l'Application}

Nous disposons des résultats d'une régression linéaire estimant le Modèle d'Évaluation des Actifs Financiers (CAPM) pour Bombardier avec 120 observations mensuelles. Les résultats de la régression sont les suivants :

\begin{landscape}
\begin{table}[h!]
    \centering
    \begin{tabular}{lcccc}
        \toprule
        \textbf{Call:} & \multicolumn{4}{c}{\texttt{lm(formula = r\_b \textasciitilde{} r\_m, data = data)}} \\
        \midrule
        \textbf{Residuals:} & \multicolumn{4}{c}{Min -0.25289, 1Q -0.09023, Median -0.02297, 3Q 0.07237, Max 0.38469} \\
        \midrule
        \textbf{Coefficients:} & \textbf{Estimate} & \textbf{Std. Error} & \textbf{t value} & \textbf{Pr(>|t|)} \\
        \midrule
        (Intercept) & -0.15834 & 0.01213 & -13.057 & $<2\times10^{-16}$\textsuperscript{***} \\
        r\_m & 1.87996 & 0.65239 & 2.882 & 0.0047\textsuperscript{**} \\
        \bottomrule
    \end{tabular}
    \caption{Résultats de la régression CAPM pour Bombardier}
\end{table}
\end{landscape}

\noindent \textbf{Note :} Les codes de significativité sont les suivants : $^{***}$ p $< 0.001$, $^{**}$ p $< 0.01$, $^{*}$ p $< 0.05$.

\subsection{Objectifs des Tests}

Réaliser les tests suivants à un niveau de significativité de 5\% :

\begin{enumerate}
    \item Tester, à l'aide d'une statistique de Student, si l'intercept (\(\beta_0\)) est nul.
    \item Tester si l'intercept (\(\beta_0\)) est plus grand ou égal à 0.
    \item Tester si le coefficient bêta de Bombardier (\(\beta_1\)) est plus petit ou égal au bêta du marché (\(\beta_m = 1\)).
    \item Tester si le coefficient bêta de Bombardier (\(\beta_1\)) est égal au bêta du marché (\(\beta_m = 1\)).
    \item Tester si le coefficient bêta de Bombardier (\(\beta_1\)) est plus grand ou égal au bêta du marché (\(\beta_m = 1\)).
\end{enumerate}

\subsection{Résultats des Tests}

Les résultats des tests d'hypothèses sont présentés dans le tableau suivant :

\begin{landscape}
\begin{table}[h!]
    \centering
    \begin{tabular}{@{}llll@{}}
        \toprule
        \textbf{Test} & \textbf{Statistique t} & \textbf{Valeur Critique} & \textbf{Décision} \\ \midrule
        a) Bilatéral : \(\beta_0 = 0\) & -13.057 & 1.980 & Rejeter \(H_0\) \\
        b) Unilatéral gauche : \(\beta_0 \geq 0\) & -13.057 & -1.658 & Rejeter \(H_0\) \\
        c) Unilatéral droite : \(\beta_1 \leq 1\) & 1.349 & 1.658 & Ne pas rejeter \(H_0\) \\
        d) Bilatéral : \(\beta_1 = 1\) & 1.349 & 1.980 & Ne pas rejeter \(H_0\) \\
        e) Unilatéral gauche : \(\beta_1 \geq 1\) & 1.349 & -1.658 & Ne pas rejeter \(H_0\) \\ \bottomrule
    \end{tabular}
    \caption{Résumé des tests d'hypothèses réalisés}
\end{table}
\end{landscape}

\subsection{Interprétation Économique}

Les résultats indiquent que :

\begin{itemize}
    \item \textbf{Intercept (\(\beta_0\))} :
    \begin{itemize}
        \item \textbf{Test a)} : Nous rejetons \(H_0\) bilatéralement, ce qui suggère que l'intercept est significativement différent de zéro.
        \item \textbf{Test b)} : Nous rejetons \(H_0\) unilatéralement, indiquant que l'intercept est significativement inférieur à zéro.
    \end{itemize}
    \item \textbf{Bêta (\(\beta_1\))} :
    \begin{itemize}
        \item \textbf{Test c)} : Nous ne rejetons pas \(H_0\) unilatéralement, ce qui suggère que \(\beta_1\) n'est pas significativement supérieur à 1.
        \item \textbf{Test d)} : Nous ne rejetons pas \(H_0\) bilatéralement, indiquant que \(\beta_1\) n'est pas significativement différent de 1.
        \item \textbf{Test e)} : Nous ne rejetons pas \(H_0\) unilatéralement, ce qui suggère que \(\beta_1\) n'est pas significativement inférieur à 1.
    \end{itemize}
\end{itemize}

Ces résultats suggèrent que le modèle CAPM estimé pour Bombardier présente un intercept négatif significatif, mais le coefficient bêta n'est pas significativement différent de celui du marché (\(\beta_m = 1\)). Cela indique que Bombardier présente une sensibilité similaire aux mouvements du marché, conformément aux attentes du modèle CAPM.

% ------------------------------------------------------------------------
% SECTION 3 : Solutions
% ------------------------------------------------------------------------
\section*{Solutions}
\addcontentsline{toc}{section}{Solutions}

\subsection*{a) Test bilatéral sur l'intercept (\(\beta_0 = 0\))}
\[
H_0 : \beta_0 = 0 \quad \text{contre} \quad H_1 : \beta_0 \neq 0
\]

\subsubsection*{Calcul de la Statistique de Test}
\[
t = \frac{\hat{\beta}_0 - 0}{SE(\hat{\beta}_0)} = \frac{-0.15834}{0.01213} \approx -13.057
\]

\subsubsection*{Détermination de la Valeur Critique}
Pour un test bilatéral avec un niveau de significativité de 5\% et 118 degrés de liberté (\(df = 118\)), la valeur critique est approximativement 1.980.

\subsubsection*{Prise de Décision}
\[
|t| = 13.057 > 1.980 \Rightarrow \text{Rejeter } H_0
\]

\text{Conclusion : L'intercept est significativement différent de zéro.}

\subsection*{b) Test unilatéral que l’intercept est plus grand ou égal à 0 (\(\beta_0 \geq 0\))}
\[
H_0 : \beta_0 \geq 0 \quad \text{contre} \quad H_1 : \beta_0 < 0
\]

\subsubsection*{Calcul de la Statistique de Test}
\[
t = \frac{\hat{\beta}_0 - 0}{SE(\hat{\beta}_0)} = \frac{-0.15834}{0.01213} \approx -13.057
\]

\subsubsection*{Détermination de la Valeur Critique}
Pour un test unilatéral à gauche avec \(\alpha = 5\%\) et 118 degrés de liberté, la valeur critique est environ -1.658.

\subsubsection*{Prise de Décision}
\[
t = -13.057 < -1.658 \Rightarrow \text{Rejeter } H_0
\]

\textbf{Conclusion} : L'intercept est significativement inférieur à zéro.

\subsection*{c) Test unilatéral pour tester si \(\beta_1 \leq 1\)}
\[
H_0 : \beta_1 \leq 1 \quad \text{contre} \quad H_1 : \beta_1 > 1
\]

\subsubsection*{Calcul de la Statistique de Test}
\[
t = \frac{\hat{\beta}_1 - 1}{SE(\hat{\beta}_1)} = \frac{1.87996 - 1}{0.65239} \approx 1.349
\]

\subsubsection*{Détermination de la Valeur Critique}
Pour un test unilatéral à droite avec \(\alpha = 5\%\) et 118 degrés de liberté, la valeur critique est environ 1.658.

\subsubsection*{Prise de Décision}
\[
t = 1.349 < 1.658 \Rightarrow \text{Ne pas rejeter } H_0
\]

\textbf{Conclusion} : Il n'y a pas suffisamment de preuves pour conclure que le bêta de Bombardier est supérieur à 1.

\subsection*{d) Test bilatéral pour tester si \(\beta_1 = 1\)}
\[
H_0 : \beta_1 = 1 \quad \text{contre} \quad H_1 : \beta_1 \neq 1
\]

\subsubsection*{Calcul de la Statistique de Test}
\[
t = \frac{\hat{\beta}_1 - 1}{SE(\hat{\beta}_1)} = \frac{1.87996 - 1}{0.65239} \approx 1.349
\]

\subsubsection*{Détermination de la Valeur Critique}
Pour un test bilatéral avec \(\alpha = 5\%\) et 118 degrés de liberté, la valeur critique est environ 1.980.

\subsubsection*{Prise de Décision}
\[
|t| = 1.349 < 1.980 \Rightarrow \text{Ne pas rejeter } H_0
\]

\textbf{Conclusion} : Il n'y a pas suffisamment de preuves pour conclure que le bêta de Bombardier est différent de 1.

\subsection*{e) Test unilatéral pour tester si \(\beta_1 \geq 1\)}
\[
H_0 : \beta_1 \geq 1 \quad \text{contre} \quad H_1 : \beta_1 < 1
\]

\subsubsection*{Calcul de la Statistique de Test}
\[
t = \frac{\hat{\beta}_1 - 1}{SE(\hat{\beta}_1)} = \frac{1.87996 - 1}{0.65239} \approx 1.349
\]

\subsubsection*{Détermination de la Valeur Critique}
Pour un test unilatéral à gauche avec \(\alpha = 5\%\) et 118 degrés de liberté, la valeur critique est environ -1.658.

\subsubsection*{Prise de Décision}
\[
t = 1.349 > -1.658 \Rightarrow \text{Ne pas rejeter } H_0
\]

\textbf{Conclusion} : Il n'y a pas suffisamment de preuves pour conclure que le bêta de Bombardier est inférieur à 1.

% ------------------------------------------------------------------------
% SECTION : Bibliographie
% ------------------------------------------------------------------------
\section*{Bibliographie}
\addcontentsline{toc}{section}{Bibliographie}

\begin{thebibliography}{9}

\bibitem{GujaratiPorter}
Gujarati, D. N., \& Porter, D. C. (2009). \textit{Basic Econometrics}. McGraw-Hill.
  
\bibitem{Wooldridge}
Wooldridge, J. M. (2010). \textit{Econometric Analysis of Cross Section and Panel Data}. MIT Press.
  
\bibitem{Greene}
Greene, W. H. (2012). \textit{Econometric Analysis}. Pearson.

\bibitem{StockWatson}
Stock, J. H., \& Watson, M. W. (2015). \textit{Introduction to Econometrics}. Pearson.
  
\bibitem{Hamilton}
Hamilton, J. D. (1994). \textit{Time Series Analysis}. Princeton University Press.
  
\end{thebibliography}

\end{document}
