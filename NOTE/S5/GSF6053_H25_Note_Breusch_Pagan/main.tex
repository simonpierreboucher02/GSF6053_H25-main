\documentclass[14pt]{extarticle} % Utilisation de extarticle pour supporter 14pt

% ------------------------------------------------------------------------
% Packages indispensables et recommandés
% ------------------------------------------------------------------------
\usepackage[utf8]{inputenc}     % Encodage des caractères
\usepackage[T1]{fontenc}        % Encodage de la police
\usepackage[french]{babel}      % Support de la langue française
\usepackage{amsmath, amssymb}   % Environnements mathématiques enrichis
\usepackage{amsthm}             % Environnements théorèmes
\usepackage{graphicx}           % Inclusion d'images
\usepackage{hyperref}           % Liens hypertextes
\usepackage{geometry}           % Gestion des marges
\usepackage{titlesec}           % Personnalisation des titres
\usepackage{setspace}           % Gestion de l'interligne
\usepackage{xcolor}             % Gestion des couleurs
\usepackage{booktabs}           % Pour améliorer les tableaux
\usepackage{fancyhdr}           % Personnalisation des en-têtes et pieds de page
\usepackage{cleveref}           % Références intelligentes
\usepackage{caption}            % Pour personnaliser les légendes
\usepackage{enumitem}           % Pour personnaliser les listes
\usepackage{pdflscape}          % Pour placer les tableaux en paysage

% ------------------------------------------------------------------------
% Configuration de la mise en page
% ------------------------------------------------------------------------
\geometry{
    a4paper,
    margin=25mm
}

% ------------------------------------------------------------------------
% Définition des environnements théorèmes, définitions, etc.
% ------------------------------------------------------------------------
\theoremstyle{definition}
\newtheorem{definition}{Définition}[section]

\theoremstyle{plain}
\newtheorem{proposition}{Proposition}[section]
\newtheorem{theorem}{Théorème}[section]

% ------------------------------------------------------------------------
% Personnalisation des sections
% ------------------------------------------------------------------------
\titleformat{\section}{\Large\bfseries}{\thesection}{1em}{}
\titleformat{\subsection}{\large\bfseries}{\thesubsection}{1em}{}
\titleformat{\subsubsection}{\normalsize\bfseries}{\thesubsubsection}{1em}{}

% ------------------------------------------------------------------------
% Commande pour mettre en avant les livres en bleu
% ------------------------------------------------------------------------
\newcommand{\livre}[1]{\textcolor{blue}{#1}}

% ------------------------------------------------------------------------
% Configuration des hyperliens
% ------------------------------------------------------------------------
\hypersetup{
    colorlinks=true,          % Les liens sont colorés
    linkcolor=blue,           % Couleur des liens internes (TOC, références, etc.)
    urlcolor=blue,            % Couleur des URL
    citecolor=blue,           % Couleur des citations
    filecolor=blue,           % Couleur des liens vers des fichiers
    pdfborder={0 0 0},        % Pas de bordure autour des liens
    breaklinks=true            % Permet les sauts de ligne dans les liens
}

% ------------------------------------------------------------------------
% Configuration des en-têtes et pieds de page
% ------------------------------------------------------------------------
\pagestyle{fancy}
\fancyhf{}
\fancyhead[L]{Test de Breusch-Pagan}
\fancyhead[R]{Hiver 2025}
\fancyfoot[C]{\thepage}
\fancyfoot[R]{\includegraphics[height=1cm]{logo_universite_laval.png}} % Ajout du logo ERS en bas à droite

\title{Le Test de Breusch-Pagan pour l'Hétéroscédasticité}
\author{GSF-6053}
\date{Hiver 2025}

\begin{document}

\maketitle
\tableofcontents
\newpage

\section{Introduction}

Le test de \textbf{Breusch-Pagan} est une procédure statistique utilisée pour détecter la présence d'hétéroscédasticité dans un modèle de régression linéaire. L'hétéroscédasticité se produit lorsque la variance des erreurs n'est pas constante à travers les observations, ce qui viole l'une des hypothèses fondamentales du modèle de régression linéaire classique.

\section{Hypothèses du Test de Breusch-Pagan}

Le test de Breusch-Pagan évalue les hypothèses suivantes :

\subsection{Hypothèse Nulle ($H_0$)}
\begin{itemize}
    \item Il n'y a pas d'hétéroscédasticité dans le modèle. La variance des erreurs est constante, c'est-à-dire $\text{Var}(u_i) = \sigma^2$ pour tout $i$.
\end{itemize}

\subsection{Hypothèse Alternative ($H_1$)}
\begin{itemize}
    \item Il existe une relation systématique entre la variance des erreurs et les variables explicatives. La variance des erreurs varie en fonction des variables explicatives, c'est-à-dire $\text{Var}(u_i) = \sigma^2 h(X_i)$ où $h(X_i)$ est une fonction des variables explicatives.
\end{itemize}

\section{Procédure du Test}

Le test de Breusch-Pagan se déroule en plusieurs étapes :

\subsection{Estimation du Modèle de Régression Principal}

On estime le modèle de régression linéaire :
\begin{align*}
    Y_i = \beta_0 + \beta_1 X_{1i} + \beta_2 X_{2i} + \dots + \beta_k X_{ki} + u_i
\end{align*}
en utilisant la méthode des moindres carrés ordinaires (MCO).

\subsection{Calcul des Résidus}

On calcule les résidus standardisés :
\begin{align*}
    \hat{u}_i = Y_i - \hat{Y}_i
\end{align*}
où $\hat{Y}_i$ est la valeur prédite par le modèle.

\subsection{Régression Auxiliaire}

On effectue une régression auxiliaire où la variable dépendante est le carré des résidus :
\begin{align*}
    \hat{u}_i^2 = \gamma_0 + \gamma_1 Z_{1i} + \gamma_2 Z_{2i} + \dots + \gamma_m Z_{mi} + \epsilon_i
\end{align*}
Les variables $Z_{1i}, Z_{2i}, \dots, Z_{mi}$ sont choisies parmi les variables explicatives originales ou d'autres transformations de celles-ci.

\subsection{Calcul de la Statistique de Test}

La statistique de test de Breusch-Pagan est donnée par :
\begin{align*}
    \text{BP} = \frac{SSR}{2\sigma^4}
\end{align*}
où $SSR$ est la somme des carrés des résidus de la régression auxiliaire.

Cependant, une approche plus courante consiste à utiliser la statistique basée sur le coefficient de détermination ($R^2$) de la régression auxiliaire :
\begin{align*}
    \text{BP} = \frac{T \cdot R^2}{2}
\end{align*}
où $T$ est le nombre total d'observations.

\subsection{Distribution Sous l'Hypothèse Nulle}

Sous l'hypothèse nulle d'homoscédasticité, la statistique de test suit une distribution du chi carré avec $m$ degrés de liberté, où $m$ est le nombre de variables explicatives dans la régression auxiliaire.

\section{Interprétation des Résultats}

\subsection{Décision Statistique}

\begin{itemize}
    \item Si $\text{BP} > \chi^2_{m, 1-\alpha}$, on rejette l'hypothèse nulle et conclut à la présence d'hétéroscédasticité.
    \item Sinon, on ne rejette pas l'hypothèse nulle.
\end{itemize}

\subsection{Conséquences de l'Hétéroscédasticité}

L'hétéroscédasticité peut entraîner :
\begin{itemize}
    \item Des estimateurs MCO toujours sans biais, mais inefficaces.
    \item Des erreurs standard incorrectes, ce qui affecte les tests d'hypothèses et les intervalles de confiance.
\end{itemize}

\section{Exemple Illustratif}

Considérons un modèle de régression où l'on souhaite estimer la relation entre le revenu ($Y$) et le niveau d'éducation ($X_1$) et l'expérience professionnelle ($X_2$).

\subsection{Étape 1 : Estimation du Modèle Principal}

On estime le modèle :
\begin{align*}
    Y_i = \beta_0 + \beta_1 X_{1i} + \beta_2 X_{2i} + u_i
\end{align*}
en utilisant les MCO.

\subsection{Étape 2 : Calcul des Résidus}

On calcule les résidus $\hat{u}_i = Y_i - \hat{Y}_i$.

\subsection{Étape 3 : Régression Auxiliaire}

On effectue la régression :
\begin{align*}
    \hat{u}_i^2 = \gamma_0 + \gamma_1 X_{1i} + \gamma_2 X_{2i} + \epsilon_i
\end{align*}

\subsection{Étape 4 : Calcul de la Statistique BP}

Supposons que $R^2 = 0.05$ et que $T = 100$ observations avec $m = 2$ variables explicatives dans la régression auxiliaire. Alors :
\begin{align*}
    \text{BP} = \frac{100 \times 0.05}{2} = 2.5
\end{align*}

\subsection{Étape 5 : Décision}

Avec $m = 2$ et un niveau de signification $\alpha = 0.05$, la valeur critique est $\chi^2_{2, 0.95} \approx 5.99$. Puisque $2.5 < 5.99$, on ne rejette pas l'hypothèse nulle d'homoscédasticité.

\section{Conclusion}

Le test de Breusch-Pagan est un outil essentiel pour détecter l'hétéroscédasticité dans les modèles de régression linéaire. En identifiant la présence d'hétéroscédasticité, les chercheurs peuvent ajuster leurs modèles en conséquence, par exemple en utilisant des estimateurs robustes ou en transformant les variables, afin d'obtenir des inférences statistiques fiables.

\section{Références}

\begin{itemize}
    \item Breusch, T. S., \& Pagan, A. R. (1979). A Simple Test for Heteroscedasticity and Random Coefficient Variation. \textit{Econometrica}, 47(5), 1287–1294.
    \item Gujarati, D. N. (2003). \textit{Basic Econometrics}. McGraw-Hill.
    \item Wooldridge, J. M. (2010). \textit{Econometric Analysis of Cross Section and Panel Data}. MIT Press.
\end{itemize}

\end{document}
