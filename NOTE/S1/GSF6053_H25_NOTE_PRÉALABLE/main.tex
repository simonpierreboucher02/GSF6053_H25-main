\documentclass[14pt]{extarticle} % Utilisation de extarticle pour supporter 14pt

% ------------------------------------------------------------------------
% Packages indispensables et recommandés
% ------------------------------------------------------------------------
\usepackage[utf8]{inputenc}     % Encodage des caractères
\usepackage[T1]{fontenc}        % Encodage de la police
\usepackage[french]{babel}      % Support de la langue française
\usepackage{amsmath, amssymb}   % Environnement mathématique enrichi
\usepackage{graphicx}           % Inclusion d'images
\usepackage{hyperref}           % Liens hypertextes
\usepackage{geometry}           % Gestion des marges
\usepackage{titlesec}           % Personnalisation des titres
\usepackage{setspace}           % Gestion de l'interligne
\usepackage{xcolor}             % Gestion des couleurs
\usepackage{lipsum}             % (optionnel) Pour générer du faux texte
\usepackage{booktabs}           % (optionnel) Pour améliorer les tableaux
\usepackage{fancyhdr}           % Personnalisation des en-têtes et pieds de page
\usepackage{cleveref}           % Références intelligentes
\usepackage{caption}            % Pour personnaliser les légendes

% ------------------------------------------------------------------------
% Configuration de la mise en page
% ------------------------------------------------------------------------
\geometry{
    a4paper,
    left=25mm,
    right=25mm,
    top=25mm,
    bottom=25mm
}

% ------------------------------------------------------------------------
% Personnalisation des sections
% ------------------------------------------------------------------------
\titleformat{\section}{\Large\bfseries}{\thesection}{1em}{}
\titleformat{\subsection}{\large\bfseries}{\thesubsection}{1em}{}
\titleformat{\subsubsection}{\normalsize\bfseries}{\thesubsubsection}{1em}{}

% ------------------------------------------------------------------------
% Commande pour mettre en avant les livres en bleu
% ------------------------------------------------------------------------
\newcommand{\livre}[1]{\textcolor{blue}{#1}}

% ------------------------------------------------------------------------
% Configuration des hyperliens
% ------------------------------------------------------------------------
\hypersetup{
    colorlinks=true,          % Les liens sont colorés
    linkcolor=blue,           % Couleur des liens internes (TOC, références, etc.)
    urlcolor=blue,            % Couleur des URL
    citecolor=blue,           % Couleur des citations
    filecolor=blue,           % Couleur des liens vers des fichiers
    pdfborder={0 0 0},        % Pas de bordure autour des liens
    breaklinks=true            % Permet les sauts de ligne dans les liens
}

% ------------------------------------------------------------------------
% Configuration des en-têtes et pieds de page
% ------------------------------------------------------------------------
\pagestyle{fancy}
\fancyhf{}
\fancyhead[L]{Préalable}
\fancyhead[R]{Hiver 2025}
\fancyfoot[C]{\thepage}
\fancyfoot[R]{\includegraphics[height=1cm]{logo_universite_laval.png}} % Ajout du logo ERS en bas à droite

% ------------------------------------------------------------------------
% Informations sur le document
% ------------------------------------------------------------------------
\title{\textbf{Préalable}}
\author{GSF-6053}
\date{Hiver 2025}

% ------------------------------------------------------------------------
% Début du document
% ------------------------------------------------------------------------
\begin{document}

\maketitle
\tableofcontents
\newpage

% Augmenter l'interligne à 1,5
\onehalfspacing

% ------------------------------------------------------------------------
% SECTION 1 : Préalable
% ------------------------------------------------------------------------
\section{Préalable}

Ce document présente une révision des concepts mathématiques de base utiles au cours. 
L’idée de cette révision n’est pas de voir toutes les propriétés présentées dans les annexes, 
mais de repasser certaines qui pourraient vous être utiles dans le cours. Attention donc à ne 
pas voir cette présentation comme exhaustive puisqu’il s’agit d’un résumé.

La référence pour cette révision se trouve dans votre \livre{livre de Wooldridge} (A, B et D partiellement) 
ou de \livre{Gujarati et Porter} (A et B). Les annexes C et E de \livre{Wooldridge} seront vues en cours de route. 
La consultation du \livre{livre de Greene} vous permettra d’explorer ces concepts en profondeur.

% ------------------------------------------------------------------------
% SECTION 2 : Révision de Statistique de Base
% ------------------------------------------------------------------------
\section{Révision de Statistique de Base}

\subsection{Les Propriétés des Sommations, des Doubles Sommations et des Produits}

Les constantes peuvent sortir des sommations de la façon suivante :
\begin{align*}
    \sum_{i=1}^{N} c x_i &= c \sum_{i=1}^{N} x_i
\end{align*}

Aussi, l’opérateur se distribue :
\begin{align*}
    \sum_{i=1}^{N} (x_i + y_i) &= \sum_{i=1}^{N} x_i + \sum_{i=1}^{N} y_i
\end{align*}

Lorsqu’une double sommation est introduite, les opérateurs sommation sont interchangeables :
\begin{align*}
    \sum_{i=1}^{N} \sum_{j=1}^{M} x_i y_j &= \sum_{i=1}^{N} x_i \sum_{j=1}^{M} y_j
\end{align*}

La moyenne échantillonnale est :
\begin{align*}
    \bar{x} &= \frac{1}{n} \sum_{i=1}^{n} x_i
\end{align*}

Il découle que la somme des déviations par rapport à la moyenne est nulle :
\begin{align*}
    \sum_{i=1}^{n} (x_i - \bar{x}) &= 0
\end{align*}

Et que :
\begin{align*}
    \sum_{i=1}^{n} (x_i - \bar{x})^2 &= \sum_{i=1}^{n} x_i^2 - n(\bar{x})^2
\end{align*}

Vous pouvez distribuer le produit pour vous en convaincre en prenant soin de distribuer 
l’opérateur de sommation de la bonne façon également.

\textbf{Note} : La section A.2 de \livre{Wooldridge} est une révision des fonctions linéaires. 
Comme c’est la base de la régression, je vous suggère de la lire avant la semaine prochaine. 
La section A.3 sur les proportions est également pertinente, surtout pour ceux d’entre vous 
qui commencent en économie/gestion. Finalement, on m’a demandé dans la séance synchrone 
davantage d’informations sur la transformation logarithmique. Vous avez une section à cet effet (A.4b).

\subsection{Les Fonctions de Densité}

Une fonction de densité peut être vue comme « comment je représente les probabilités que quelque 
chose arrive quand on a affaire avec une variable aléatoire ». Pour les densités, il existe deux 
types : les variables discrètes et les variables continues. Les variables discrètes ne prennent que 
des valeurs spécifiques, telles que des entiers. Les variables continues peuvent prendre n’importe 
quelle valeur.

Pour une variable discrète, c’est la probabilité qu’une variable aléatoire prenne une certaine valeur. 
Par exemple, si vous représentez la somme de deux dés, vous avez une fonction de densité discrète, 
ce qui donne une fonction de densité en bâtonnets.

Dans le cas d’une variable aléatoire continue, la fonction de densité doit répondre aux critères 
suivants :
\begin{align*}
    \int_{-\infty}^{\infty} f(x) \, dx &= 1
\end{align*}

Un exemple de fonction de densité continue serait la cloche de la loi normale.

Il est important de discerner que dans le cas d’une variable continue, la probabilité que la variable 
aléatoire prenne une valeur spécifique est nulle. Seule la possibilité que la variable aléatoire soit 
comprise à l’intérieur d’un intervalle peut être investiguée.

La fonction de densité conditionnelle est :
\begin{align*}
    f(x \mid y) &= \frac{f(x, y)}{f(y)}
\end{align*}

L’indépendance de deux variables : lorsque la densité jointe est égale au produit des densités marginales. 
Si deux événements sont indépendants, ils ne devraient pas être étudiés ensemble.

\subsection{Les Caractéristiques des Distributions de Probabilités}

\textit{\livre{Wooldridge}, B-3 à B-5}. Nous présentons ici une révision des formules. Les concepts 
sont présentés très en détail dans le livre.

\textbf{L’espérance} : Il s’agit de la valeur attendue :
\begin{align*}
    E(X) &= \sum x \, f(x)
\end{align*}

Et si les événements sont équipondérés, cela revient à :
\begin{align*}
    E(X) &= \frac{1}{n} \sum_{i=1}^{n} x_i
\end{align*}

\textbf{Propriétés importantes pour le cours} :
\begin{align*}
    E(aX + b) &= aE(X) + b \quad \text{si } a \text{ et } b \text{ sont des quantités connues.}
\end{align*}

Et :
\begin{align*}
    E(XY) &= E(X)E(Y) + \text{cov}(X, Y)
\end{align*}

\textbf{La variance} :

Supposons \(E(X) = \mu\), l’écart dans la distribution par rapport à la moyenne sera donc :
\begin{align*}
    \text{Var}(X) &= \sigma^2 = E\bigl((X - \mu)^2\bigr)
\end{align*}

\begin{align*}
    \text{Var}(aX + b) &= a^2 \text{Var}(X), \quad \text{si } a \text{ et } b \text{ sont connus.}
\end{align*}

\begin{align*}
    \text{Var}(X + Y) &= \text{Var}(X) + \text{Var}(Y) + 2 \,\text{Cov}(X, Y)
\end{align*}

\subsection{Algèbre Matricielle}

\textbf{Les matrices} :

\begin{align*}
    A &= [a_{ij}]_{M \times N}
\end{align*}
où \(i\) dénote la ligne et \(j\) dénote la colonne. Les dimensions de \(A\) sont donc 
\(M \times N\).

Les opérations de base sur les matrices sont à réviser par vous-même. Celles-ci incluent 
l’addition, la soustraction, la transposition, la multiplication et l’inversion. Portez une 
attention particulière aux propriétés de la multiplication.

Voici une propriété importante de la transposition :
\begin{align*}
    (ABCD)^{\prime} &= D^{\prime}C^{\prime}B^{\prime}A^{\prime}
\end{align*}

\textbf{Rang d’une matrice} : Le nombre de colonnes ou de lignes linéairement indépendantes. 
La matrice est de plein rang si celui-ci est égal au nombre de colonnes de la matrice. 
Une matrice est inversible seulement si elle est de plein rang colonne.

Pour les matrices inversibles :
\begin{align*}
    A A^{-1} &= I, \\
    (A^{-1})^{\prime} &= (A^{\prime})^{-1}
\end{align*}

\textbf{La trace d’une matrice} : La somme des éléments sur la diagonale d’une matrice carrée. 
Cet opérateur est intéressant pour nous, car :
\begin{align*}
    \text{tr}(AB) &= \text{tr}(BA)
\end{align*}

\textbf{Quelques matrices importantes} :
\begin{itemize}
    \item \textbf{Matrice symétrique} : Une matrice carrée telle que \(A = A^{\prime}\).
    \item \textbf{Matrice diagonale} : Contient des éléments seulement sur la diagonale et 
    des zéros sur les éléments hors diagonale.
    \item \textbf{Matrice idempotente} : Une matrice carrée qui, lorsqu’elle est multipliée 
    par elle-même, donne elle-même comme produit, c’est-à-dire \(A \cdot A = A\).
\end{itemize}

% ------------------------------------------------------------------------
% SECTION 3 : Bibliographie
% ------------------------------------------------------------------------
\section*{Bibliographie}
\addcontentsline{toc}{section}{Bibliographie}

\begin{thebibliography}{9}

\bibitem{Wooldridge}
Wooldridge, J. M. (2010). \textit{Econometric Analysis of Cross Section and Panel Data}. MIT Press.

\bibitem{GujaratiPorter}
Gujarati, D. N., \& Porter, D. C. (2009). \textit{Basic Econometrics}. McGraw-Hill.

\bibitem{Greene}
Greene, W. H. (2012). \textit{Econometric Analysis}. Pearson.

\end{thebibliography}

\end{document}
